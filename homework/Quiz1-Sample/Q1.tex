\documentclass[11pt]{article}
\usepackage{amsmath,amssymb,amsmath,amsthm,amsfonts}
\usepackage{latexsym,graphicx}
\usepackage{fullpage,color}
\usepackage{url,hyperref}
\usepackage{natbib}
\usepackage{graphicx,subfigure}
\usepackage{algorithm}
\usepackage{algorithmic}
\usepackage{listings}
\usepackage{xcolor}
\usepackage{color}

\numberwithin{equation}{section}

\pagestyle{plain}

\setlength{\oddsidemargin}{0in}
\setlength{\topmargin}{0in}
\setlength{\textwidth}{6.5in}
\setlength{\textheight}{8.5in}

\newtheorem{fact}{Fact}[section]
\newtheorem{question}{Question}[section]
\newtheorem{lemma}{Lemma}[section]
\newtheorem{theorem}[lemma]{Theorem}
\newtheorem{assumption}[lemma]{Assumption}
\newtheorem{corollary}[lemma]{Corollary}
\newtheorem{prop}[lemma]{Proposition}
\newtheorem{claim}{Claim}[section]
\newtheorem{remark}{Remark}[section]
\newtheorem{definition}{Definition}[section]
\newtheorem{prob}{Problem}[section]
\newtheorem{conjecture}{Conjecture}[section]
\newtheorem{property}{Property}[section]

\def\A{{\bf A}}
\def\a{{\bf a}}
\def\B{{\bf B}}
\def\bb{{\bf b}}
\def\C{{\bf C}}
\def\c{{\bf c}}
\def\D{{\bf D}}
\def\d{{\bf d}}
\def\E{{\bf E}}
\def\e{{\bf e}}
\def\F{{\bf F}}
\def\f{{\bf f}}
\def\g{{\bf g}}
\def\h{{\bf h}}
\def\G{{\bf G}}
\def\H{{\bf H}}
\def\I{{\bf I}}
\def\K{{\bf K}}
\def\k{{\bf k}}
\def\LL{{\bf L}}
\def\M{{\bf M}}
\def\m{{\bf m}}
\def\N{{\bf N}}
\def\n{{\bf n}}
\def\PP{{\bf P}}
\def\Q{{\bf Q}}
\def\q{{\bf q}}
\def\R{{\bf R}}
\def\rr{{\bf r}}
\def\S{{\bf S}}
\def\s{{\bf s}}
\def\T{{\bf T}}
\def\tt{{\bf t}}
\def\U{{\bf U}}
\def\u{{\bf u}}
\def\V{{\bf V}}
\def\v{{\bf v}}
\def\W{{\bf W}}
\def\w{{\bf w}}
\def\X{{\bf X}}
\def\x{{\bf x}}
\def\Y{{\bf Y}}
\def\y{{\bf y}}
\def\Z{{\bf Z}}
\def\z{{\bf z}}
\def\0{{\bf 0}}
\def\1{{\bf 1}}



\def\AM{{\mathcal A}}
\def\CM{{\mathcal C}}
\def\DM{{\mathcal D}}
\def\EM{{\mathcal E}}
\def\GM{{\mathcal G}}
\def\FM{{\mathcal F}}
\def\IM{{\mathcal I}}
\def\JM{{\mathcal J}}
\def\KM{{\mathcal K}}
\def\LM{{\mathcal L}}
\def\NM{{\mathcal N}}
\def\OM{{\mathcal O}}
\def\PM{{\mathcal P}}
\def\SM{{\mathcal S}}
\def\TM{{\mathcal T}}
\def\UM{{\mathcal U}}
\def\VM{{\mathcal V}}
\def\WM{{\mathcal W}}
\def\XM{{\mathcal X}}
\def\YM{{\mathcal Y}}
\def\RB{{\mathbb R}}
\def\RBmn{{\RB^{m\times n}}}
\def\EB{{\mathbb E}}
\def\PB{{\mathbb P}}

\def\TX{\tilde{\bf X}}
\def\TA{\tilde{\bf A}}
\def\tx{\tilde{\bf x}}
\def\ty{\tilde{\bf y}}
\def\TZ{\tilde{\bf Z}}
\def\tz{\tilde{\bf z}}
\def\hd{\hat{d}}
\def\HD{\hat{\bf D}}
\def\hx{\hat{\bf x}}
\def\nysA{{\tilde{\A}_c^{\textrm{nys}}}}

\def\alp{\mbox{\boldmath$\alpha$\unboldmath}}
\def\bet{\mbox{\boldmath$\beta$\unboldmath}}
\def\epsi{\mbox{\boldmath$\epsilon$\unboldmath}}
\def\etab{\mbox{\boldmath$\eta$\unboldmath}}
\def\ph{\mbox{\boldmath$\phi$\unboldmath}}
\def\pii{\mbox{\boldmath$\pi$\unboldmath}}
\def\Ph{\mbox{\boldmath$\Phi$\unboldmath}}
\def\Ps{\mbox{\boldmath$\Psi$\unboldmath}}
\def\ps{\mbox{\boldmath$\psi$\unboldmath}}
\def\tha{\mbox{\boldmath$\theta$\unboldmath}}
\def\Tha{\mbox{\boldmath$\Theta$\unboldmath}}
\def\muu{\mbox{\boldmath$\mu$\unboldmath}}
\def\Si{\mbox{\boldmath$\Sigma$\unboldmath}}
\def\si{\mbox{\boldmath$\sigma$\unboldmath}}
\def\Gam{\mbox{\boldmath$\Gamma$\unboldmath}}
\def\Lam{\mbox{\boldmath$\Lambda$\unboldmath}}
\def\De{\mbox{\boldmath$\Delta$\unboldmath}}
\def\Ome{\mbox{\boldmath$\Omega$\unboldmath}}
\def\Pii{\mbox{\boldmath$\Pi$\unboldmath}}
\def\varepsi{\mbox{\boldmath$\varepsilon$\unboldmath}}
\newcommand{\ti}[1]{\tilde{#1}}
\def\Ncal{\mathcal{N}}
\def\argmax{\mathop{\rm argmax}}
\def\argmin{\mathop{\rm argmin}}

\def\ALG{{\AM_{\textrm{col}}}}

\def\softmax{\mathsf{softmax}}
\def\sigmoid{\mathsf{sigmoid}}
\def\bias{\mathsf{bias}}
\def\var{\mathsf{var}}
\def\sgn{\mathsf{sgn}}
\def\tr{\mathsf{tr}}
\def\rk{\mathrm{rank}}
\def\nnz{\mathsf{nnz}}
\def\poly{\mathrm{poly}}
\def\diag{\mathsf{diag}}
\def\Diag{\mathsf{Diag}}
\def\const{\mathrm{Const}}
\def\st{\mathsf{s.t.}}
\def\vect{\mathsf{vec}}
\def\sech{\mathrm{sech}}

\newcommand{\red}[1]{{\color{red}#1}}



\def\argmax{\mathop{\rm argmax}}
\def\argmin{\mathop{\rm argmin}}

\newenvironment{note}[1]{\medskip\noindent \textbf{#1:}}%
        {\medskip}


\newcommand{\etal}{{\em et al.}\ }
\newcommand{\assign}{\leftarrow}
\newcommand{\eps}{\epsilon}

\newcommand{\opt}{\textrm{\sc OPT}}
\newcommand{\script}[1]{\mathcal{#1}}
\newcommand{\ceil}[1]{\lceil #1 \rceil}
\newcommand{\floor}[1]{\lfloor #1 \rfloor}



\lstset{ %
extendedchars=false,            % Shutdown no-ASCII compatible
language=Python,                % choose the language of the code
xleftmargin=1em,
xrightmargin=1em,
basicstyle=\footnotesize,    % the size of the fonts that are used for the code
tabsize=3,                            % sets default tabsize to 3 spaces
numbers=left,                   % where to put the line-numbers
numberstyle=\tiny,              % the size of the fonts that are used for the line-numbers
stepnumber=1,                   % the step between two line-numbers. If it's 1 each line
                                % will be numbered
numbersep=5pt,                  % how far the line-numbers are from the code   %
keywordstyle=\color[rgb]{0,0,1},                % keywords
commentstyle=\color[rgb]{0.133,0.545,0.133},    % comments
stringstyle=\color[rgb]{0.627,0.126,0.941},      % strings
backgroundcolor=\color{white}, % choose the background color. You must add \usepackage{color}
showspaces=false,               % show spaces adding particular underscores
showstringspaces=false,         % underline spaces within strings
showtabs=false,                 % show tabs within strings adding particular underscores
frame=single,                 % adds a frame around the code
%captionpos=b,                   % sets the caption-position to bottom
breaklines=true,                % sets automatic line breaking
breakatwhitespace=false,        % sets if automatic breaks should only happen at whitespace
%title=\lstname,                 % show the filename of files included with \lstinputlisting;
%                                % also try caption instead of title
mathescape=true,escapechar=?    % escape to latex with ?..?
escapeinside={\%*}{*)},         % if you want to add a comment within your code
%columns=fixed,                  % nice spacing
%morestring=[m]',                % strings
%morekeywords={%,...},%          % if you want to add more keywords to the set
%    break,case,catch,continue,elseif,else,end,for,function,global,%
%    if,otherwise,persistent,return,switch,try,while,...},%
}


\begin{document}

%\setlength{\fboxrule}{.5mm}\setlength{\fboxsep}{1.2mm}
%\newlength{\boxlength}\setlength{\boxlength}{\textwidth}
%\addtolength{\boxlength}{-4mm}


\title{CS583A: Quiz 1 (Sample Questions)}

\author{{\bf Name}: ~~~~~~~~~\qquad ~~~~~ \qquad~~~~~~~~~}

\date{ }

\maketitle


%\newpage
%
%\setcounter{tocdepth}{2}% Allow only \subsection in ToC
%
%\tableofcontents
%\newpage

%\section{Introduction}

\paragraph{Policy:}
Books and printed materials are allowed. Do not use electronic divice, including phone, laptop, and tablet.


\paragraph{Hint:} (i) $\frac{\partial  e^a}{\partial a} = e^a$, (ii) $\frac{ \partial \log_e (a) }{\partial a  } = \frac{1}{a}$, (iii) $\frac{ \partial \frac{1}{a} }{\partial a  } = - \frac{1}{a^2}$, (iv) $\frac{ \partial a^4 }{\partial a  } = 4a^3$, and (v) $\frac{ \partial \cos (a) }{\partial a  } = -\sin (a)$.


\vspace{3mm}




\paragraph{Q1 (5\%).} 
{\bf (Fill in the blank.)}
What is the output of the following Python program?\\
Answer: \underline{~\qquad\qquad\qquad~}
%ANSWER: 55

\begin{lstlisting}
import numpy
a = numpy.array([1, 2, 3, 4, 5])
b = numpy.sum(a * a)
print(b)
\end{lstlisting}
\vspace{3mm}



\paragraph{Q2 (5\%).} 
{\bf (Fill in the blank.)}
What is the output of the following Python program?\\
Answer: \underline{~\qquad\qquad\qquad~}
%ANSWER: 3



\begin{lstlisting}
import numpy
a = numpy.random.rand(6, 7) # generate a random matrix
b = numpy.random.rand(7, 8) # generate a random matrix
c = numpy.dot(a, b)
print(c.shape[1])
\end{lstlisting}
\vspace{3mm}



\paragraph{Q3 (12\%).} 
{\bf (Fill in the blanks.)}
$\a = [-1, 2, 0, 0, -3]^T$ is a vector.
Calculate the following vector norms.
\begin{itemize}
	\item 
	The squared $\ell_2$ norm: $\| \a \|_2^2 = $ \underline{~\qquad\qquad\qquad~} .
	\item 
	The $\ell_1$ norm: $\| \a \|_1 = $ \underline{~\qquad\qquad\qquad~} .
	\item 
	The $\ell_\infty$ norm: $\| \a \|_\infty = $ \underline{~\qquad\qquad\qquad~} .
\end{itemize}



\paragraph{Q4 (16\%).} 
{\bf (Fill in the blanks.)}
Let $\I_5$ be the $5\times 5$ identity matrix.
Calculate the following values.
\begin{itemize}
	\item 
	The largest eigenvalue: $\lambda_{\max} (\I_5) = $ \underline{~\qquad\qquad\qquad~} .
	\item 
	The smallest eigenvalue: $\lambda_{\min} (\I_5) = $ \underline{~\qquad\qquad\qquad~} .
	\item 
	The trace: $\tr (\I_5 ) = $ \underline{~\qquad\qquad\qquad~} .
	\item 
	The squared Frobenius norm: $\| \I_5 \|_F^2 = $ \underline{~\qquad\qquad\qquad~} .
\end{itemize}




\paragraph{Q4 (12\%).} 
{\bf (Fill in the blanks.)}
$\x = [x_1, x_2 , x_3]^T$ is a $3$-dimensional vector.
What is the derivative of 
\[
f(\x) \:=\:
e^{x_1} + x_2^4 + x_2 x_3 + 5\, \cos (x_3) 
\]
w.r.t.\ the vector $\x$?
Answer:
\begin{equation*}
\frac{\partial \, f(\x)}{\partial \, \x}
\: = \:
\left[
\vspace{2mm}
\begin{array}{c}
\vspace{2mm}
\underline{~\qquad\qquad\qquad~} \\
\vspace{2mm}
\underline{~\qquad\qquad\qquad~} \\
\underline{~\qquad\qquad\qquad~} \\
\end{array}
\right] .
\end{equation*}



%\newpage
%\bibliographystyle{plain}
%
%%\markboth{\bibname}{\bibname}
%\bibliography{matrix}


\end{document}
